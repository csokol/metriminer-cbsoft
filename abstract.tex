\documentclass[12pt]{article}

\usepackage{sbc-template}

\usepackage{graphicx,url}

\usepackage[brazil]{babel}   
\usepackage[latin1]{inputenc}  
     
\sloppy

\title{MetricMiner: uma ferramenta web de apoio � minera��o de reposit�rios de software}

\author{Francisco Zigmund Sokol\inst{1}, Mauricio Aniche\inst{1}, Marco Aur�lio Gerosa\inst{1}}


\address{Instituto de Matem�tica e Estat�stica -- Universidade de S�o Paulo
  (USP)\\
  Rua do Mat�o, 1010 -- 05508-090 -- S�o Paulo -- SP -- Brasil
  \email{chico.sokol@gmail.com, aniche@ime.usp.br, gerosa@ime.usp.br}
}

\begin{document} 

\maketitle
     
\begin{resumo} 
    Este trabalho apresenta uma ferramenta web que oferece suporte ao 
    pesquisador da �rea de minera��o de reposit�rios de software, o MetricMiner.
    Esse sistema armazena diversas informa��es de reposit�rios e calcula uma 
    s�rie de m�tricas de c�digo fonte sobre todo o hist�rico dos projetos a partir
    dos seus sistemas de controle de vers�es. Posteriormente, o pesquisador � capaz
    de extrair os dados calculados pelo sistema e realizar an�lises 
    estat�sticas sobre os resultados. 
    Como toda a intera��o � feita por meio de uma interface web, o MetricMiner 
    facilita muito o processo de pesquisa na �rea, j� que com os softwares dispon�veis
    atualmente, o pesquisador precisa configurar a ferramenta localmente e baixar os
    projetos sobre os quais deseja calcular suas m�tricas, tornando o processo dif�cil
    e muito demorado. Al�m disso, o sistema disponibiliza uma base com dados de uma
    s�rie de projetos open source sobre os quais um pesquisador pode extrair e 
    relacionar informa��es, al�m de submeter novas m�tricas para serem calculadas para
    esses projetos. Tamb�m ser� apresentada uma an�lise simples de um projeto
    de grande porte, exibindo o potencial das funcionalidades da ferramenta.
\end{resumo}


\end{document}
